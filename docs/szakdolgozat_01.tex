% !TeX spellcheck = hu_HU
\documentclass[12pt]{report}
\usepackage[utf8]{inputenc}


% Title Page
\title{Térsakk}
\author{Máté Gergely}
\date{2015 május 12.}

\begin{document}
\maketitle
\tableofcontents

\chapter{Bevezető}

A Bevezetés a témaválasztás indoklását és a megoldandó feladat rövid, közérthető leírását tartalmazza.

\chapter {A térsakkról}

\section{A térsakk rövid története}

\section{A játék szabályai, menete}


\chapter{Felhasználói dokumentáció}

A Felhasználói dokumentáció tartalmazza 
\begin{itemize}
\item a megoldott probléma rövid megfogalmazását,
\item a felhasznált módszerek rövid leírását, 
\item a program használatához szükséges összes információt.
\end{itemize}

\section{Telepítés}

\section{Futtatás}	
Parancssori paraméterek

\section{Játék egy számítógépen}
UI - a megjelenés elmagyarázása

\section{Visszajátszás}

\section{Játék hálózaton}


\chapter{Fejlesztői dokumentáció}

A Fejlesztői dokumentáció tartalmazza 
\begin{itemize}
\item a probléma részletes specifikációját, 
\item a felhasznált módszerek részletes leírását, a használt fogalmak definícióját,
\item a program logikai és fizikai szerkezetének leírását (adatszerkezetek, adatbázisok, modulfelbontás),
\item a tesztelési tervet és a tesztelés eredményeit.
\end{itemize}

\section{A szoftver specifikációja}
Felhasználói esetek diagramokkal

\section{A használt fejlesztői eszközök és módszerek}

\section{A program logikai szerkezete}

\subsection{A vezérlés}

\subsection {A játékmodell}

\subsection{A megjelenítés}

\section{A tesztelés}

doxygen?

Érdekesebb részek...

\section{Továbbfejlesztési lehetőségek}

\end{document}          
